\documentclass[../../main.tex]{subfiles}

\begin{document}
    \chapter{Introduction}
    The human brain is a remarkably complex and efficient system, capable of performing an extraordinary range of cognitive tasks with ease. Its ability to learn from experience, adapt to new environments, and process vast amounts of sensory information has long fascinated scientists and engineers alike.

    Modeling this functionality falls into the domain of computer science, where the goal is to create algorithms inspired by the brain's structure: The human brain consists of elementary computational units, neurons, which itself can be sufficiently modeled with computers, and whose interplay creates the emergent complex behavior we see. This leads to the following premise:

    \begin{premise}
        In theory, the human brain can be sufficiently modelled by a computer program.
    \end{premise}

    Over many decades scientists have been trying to make progress towards this monumental task of modelling the human brain. As a result, two paradigms on how to implement intelligent models emerged, they are called \emph{connectionism} and \emph{symbolism}. While connectionist models rely on training data, symbolist models have all their inference rules and algorithms directly coded into them. There also exist hybrid models, summarized under the term \emph{neuro-symbolic AI}. Marcus claims:

    \begin{citecallout}[\textcite{marcus_alphageometry}]
        There can be no path to AGI without neurosymbolic AI.
    \end{citecallout}

    To understand why Marcus is so critical of connectionist models which include LLMs, we will analyze one of his early writings, namely \emph{The Algebraic Mind} \cite{marcus_algebraic_mind}. In this book, he argues that the popular architecture of MLPs (multi-layer perceptrons) is not sufficient to model the human brain, because it cannot represent certain algebraic structures. Even as of today, Marcus likes to refer to his early work, which emphasizes its importance:

    \begin{citecallout}[\textcite{marcus_alphageometry}]
        You would think the need for this reconciliation is obvious, and indeed it was the central point of my 2001 book The Algebraic Mind, which in the words of the subtitle sought to integrate connectionism (neural networks) with the cognitive science of symbol-manipulation:
    \end{citecallout}

    In this essay, we will analyze Marcus's critique of connectionist models and his arguments against the MLP architecture, with a focus on chapter 3 \textit{Relations between Variables}. We will also discuss the relevance of his critique in contemporary artificial intelligence research, especially in the context of LLMs.  
\end{document}